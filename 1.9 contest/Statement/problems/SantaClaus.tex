\begin{problem}{Santa Claus}{标准输入}{标准输出}{1 s}{256 MB}
给定一个 $H \times W$ 的网格。用 $(i,j)$ 表示第 $i$ 行第 $j$ 列的单元格。

网格中的每个单元格 $S_{i,j}$ 可能是以下三种字符之一:
\begin{itemize}
\item `\#`: 表示该单元格不可通行
\item `.`: 表示该单元格可通行且没有房屋
\item `@`: 表示该单元格可通行且有一座房屋
\end{itemize}

圣诞老人最初位于单元格 $(X,Y)$。他将按照字符串 $T$ 的指示移动。具体规则如下:

令 $|T|$ 表示字符串 $T$ 的长度。对于 $i=1,2,\ldots,|T|$,圣诞老人将按以下规则移动:

假设他当前位于单元格 $(x,y)$,则:
\begin{itemize}
\item 若 $T_i$ 为 `U` 且单元格 $(x-1,y)$ 可通行,则移动至 $(x-1,y)$
\item 若 $T_i$ 为 `D` 且单元格 $(x+1,y)$ 可通行,则移动至 $(x+1,y)$
\item 若 $T_i$ 为 `L` 且单元格 $(x,y-1)$ 可通行,则移动至 $(x,y-1)$
\item 若 $T_i$ 为 `R` 且单元格 $(x,y+1)$ 可通行,则移动至 $(x,y+1)$
\item 否则保持在原地 $(x,y)$
\end{itemize}

请计算:
\begin{enumerate}
\item 圣诞老人完成所有移动后所在的单元格坐标
\item 他在移动过程中经过的不同房屋数量(同一房屋多次经过只计一次)
\end{enumerate}

\InputFile
输入格式如下:

第一行包含四个整数 $H$、$W$、$X$、$Y$ $(1 \leq H,W \leq 50, 1 \leq X \leq H, 1 \leq Y \leq W)$,分别表示网格的行数、列数以及圣诞老人的初始位置坐标。

接下来 $H$ 行,每行包含 $W$ 个字符,描述网格。第 $i$ 行第 $j$ 个字符为 $S_{i,j}$。

最后一行包含一个由字符 `U`、`D`、`L`、`R` 组成的字符串 $T$ $(1 \leq |T| \leq 100)$,表示圣诞老人的移动指令。

\OutputFile
设 $(X,Y)$ 为圣诞老人完成所有移动后所在的单元格坐标,$C$ 为他在移动过程中经过或到达的不同房屋数量。请按顺序输出三个整数 $X$、$Y$、$C$,用空格分隔。

\Example

\begin{example}
\exmp{
5 5 3 4
\#\#\#\#\#
\#...\#
\#.@.\#
\#..@\#
\#\#\#\#\#
LLLDRUU
}{
2 3 1
} %
\exmp{
6 13 4 6
\#\#\#\#\#\#\#\#\#\#\#\#\#
\#@@@@@@@@@@@\#
\#@@@@@@@@@@@\#
\#@@@@.@@@@@@\#
\#@@@@@@@@@@@\#
\#\#\#\#\#\#\#\#\#\#\#\#\#
UURUURLRLUUDDURDURRR
}{
3 11 11
} %
\end{example}

\Constraints
\begin{itemize}
\item $3 \leq H,W \leq 100$
\item $1 \leq X \leq H$
\item $1 \leq Y \leq W$
\item 所有给定的数字均为整数。
\item 每个 $S_{i,j}$ 是 `\#`、`.` 或 `@` 中的一个。
\item 对于每个 $1 \leq i \leq H$,$S_{i,1}$ 和 $S_{i,W}$ 都是 `\#`。
\item 对于每个 $1 \leq j \leq W$,$S_{1,j}$ 和 $S_{H,j}$ 都是 `\#`。
\item $S_{X,Y}=$ `.`
\item $T$ 是一个长度至少为 $1$ 且最多为 $10^4$ 的字符串,由 `U`、`D`、`L`、`R` 组成。
\end{itemize}

\end{problem}
